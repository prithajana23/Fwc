\documentclass[12pt]{article}
\usepackage{gensymb}
\usepackage{amsmath}
\usepackage{float}
\let\vec\mathbf
\usepackage{float}
\providecommand{\brak}[1]{\ensuremath{\left(#1\right)}}
\providecommand{\myvec}[1]{\ensuremath{\begin{pmatrix}#1\end{pmatrix}}}
\providecommand{\norm}[1]{\ensuremath{\lvert|#1\rvert|}}
\usepackage{graphics}
\usepackage{graphicx}
\graphicspath{{storage/self/primary/Download/asgnt8/fig}}
\graphicspath{{storage/self/primary/Download/asgnt8/table}}
\usepackage{amsmath}
\usepackage{float}
\let\vec\mathbf
\usepackage{float}
\providecommand{\brak}[1]{\ensuremath{\left(#1\right)}}
\providecommand{\myvec}[1]{\ensuremath{\begin{pmatrix}#1\end{pmatrix}}}
\providecommand{\norm}[1]{\ensuremath{\lvert|#1\rvert|}}
\begin{document}
\title{\textbf{11.11.1.13}}
\date{}
\maketitle
\textbf{Question :} Find the equation of the circle with radius 5 whose centre lies on $x$-axis and passes through the point $\brak{2,3}$.

\textbf{Solution :}
\begin{table}[H]
    \centering
    \begin{tabular}{|c|c|c|}
 \hline
    \textbf{Input Parameters} &\textbf{Description} &\textbf{Value} \\
    \hline
     $\vec{A}$& Vertex(at origin)&$\vec{0}$\\
     \hline
	$a$& Side of the parallelogram &$AB \brak{= DC = 6 unit}$\\
     \hline
	$b$& Side of the parallelogram &$AD\brak{= BC = \sqrt{29} unit}$\\
     \hline
	$\theta$&Angle of parallelogram & $\angle BAD\brak{= \sin^{-1}\brak{\frac{5}{\sqrt{29}}}}$\\
     \hline
      $k_1:1$&Ratio by which $\vec{Q}$ divides $AD$  & $AQ:QD$\\
     \hline  
      $k_2:1$&Ratio by which $\vec{P}$ divides $DC$  & $DP:PC$\\
     \hline  
\end{tabular}

        \caption{Table of input parameters}
    \label{tab:11.11.1.13}
\end{table}
The general formula of the circle is
\begin{align}
\norm{\vec{x}}^2 + 2\vec{u}^{\top}\vec{x}+f&=0\\
 where,   \vec{u}=-x\vec{e_1}\\
 OA&=r\\
 \sqrt{\brak{2-x}^2+9}&=5\\
 x&=6\\
 or,x&=-2
 \end{align}
 For $x=6$
 \begin{align}
 \norm{\vec{A}}^2 + 2\vec{u}^{\top}\vec{A}+f&=0\\
 or,f&=11
\end{align}
 For $x=-2$
 \begin{align}
 \norm{\vec{A}}^2 + 2\vec{u}^{\top}\vec{A}+f&=0\\
 or,f&=-21
\end{align}
Therefore the equations of the circle are
\begin{align}
   \norm{\vec{x}}^2 - 2\myvec{6&0}\vec{x}+11&=0\\
   \norm{\vec{x}}^2 - 2\myvec{-2&0}\vec{x}-21&=0
\end{align}    
\begin{figure}[H]
    \centering
\includegraphics[width=\columnwidth]{fig/11.11.1.13.png}
    \caption{}
    \label{fig:11.11.1.13}
\end{figure}

\begin{figure}[H]
\centering
\includegraphics[width=\columnwidth]{fig/11.11.1.13.1.png}
\caption{}
\label{fig:11.11.1.13.1}
\end{figure}
\end{document}


