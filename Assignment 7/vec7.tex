\documentclass[12pt]{article}
\usepackage{gensymb}
\usepackage{amsmath}
\let\vec\mathbf
\usepackage{float}
\usepackage{graphics}
\usepackage{graphicx}
\providecommand{\brak}[1]{\ensuremath{\left(#1\right)}}
\providecommand{\myvec}[1]{\ensuremath{\begin{pmatrix}#1\end{pmatrix}}}
\providecommand{\norm}[1]{\ensuremath{\lvert|#1\rvert|}}
\graphicspath{{storage/self/primary/Download/asgnt7/table}}
\graphicspath{{storage/self/primary/Download/asgnt7/fig}}
\begin{document}
\title{\textbf{10.10.2.4}}
\date{}
\maketitle
\textbf{Question :} Prove that the tangents drawn at the ends of a diameter of a circle are parallel.

\textbf{Solution :}
\begin{table}[H]
    \centering
    \begin{tabular}{|c|c|c|}
 \hline
    \textbf{Input Parameters} &\textbf{Description} &\textbf{Value} \\
    \hline
     $\vec{A}$& Vertex(at origin)&$\vec{0}$\\
     \hline
	$a$& Side of the parallelogram &$AB \brak{= DC = 6 unit}$\\
     \hline
	$b$& Side of the parallelogram &$AD\brak{= BC = \sqrt{29} unit}$\\
     \hline
	$\theta$&Angle of parallelogram & $\angle BAD\brak{= \sin^{-1}\brak{\frac{5}{\sqrt{29}}}}$\\
     \hline
      $k_1:1$&Ratio by which $\vec{Q}$ divides $AD$  & $AQ:QD$\\
     \hline  
      $k_2:1$&Ratio by which $\vec{P}$ divides $DC$  & $DP:PC$\\
     \hline  
\end{tabular}

    \caption{Table of input parameters}
    \label{tab:11.11.1.13.1}
\end{table}
\begin{table}[H]
    \centering
    \begin{tabular}{|c|c|}
\hline
    \textbf{Output Parameters} &\textbf{Value} \\
    \hline
     $\vec{B}$&$a\begin{pmatrix}
         1\\0
     \end{pmatrix}$\\
     \hline
     $\vec{D}$&$b\begin{pmatrix}
         \cos{\theta}\\\sin{\theta}
     \end{pmatrix}$\\
     \hline
     $\vec{C}$&$\vec{B}+\vec{D}$\\
     \hline
      $\vec{ Q}$ & $\frac{k_1.\vec{D}+\vec{A}}{k_1+1}$\\
     \hline
      $\vec{P}$ &$\frac{k_2.\vec{C}+\vec{D}}{k_2+1}$\\
   \hline
\end{tabular}


    \caption{Table of output parameters}
    \label{tab:11.11.1.13.2}
\end{table}
To find the angle $\theta_2$,
\begin{align}
    OA &=OB\\
    \theta_2- \theta_1&= 180\degree\\
    \theta_2 &= 307\degree
    \end{align}
Angle between these two tangents is
\begin{align}
    \cos{\theta}&=\frac{\vec{m_1}^{\top}\vec{m_2}}{\vec{\norm{m_1}\norm{m_2}}}\\
    or,\theta &= \pi
\end{align}
Therefore,the two tangents are parallel to each other.
\begin{figure}[H]                          
\centering                    
\includegraphics[width=\columnwidth]{fig/10.10.2.4.png}                      
\caption{}              
\label{10.10.2.4}
\end{figure}
\end{document}

