\documentclass[12pt,A4 paper]{article}
\usepackage{tikz}
\usepackage{graphicx}
\usepackage{graphics}
\usepackage{listings}
\usepackage{subfig}
\usepackage{float}
\graphicspath{{/storage/self/primary/Download/latex1/image}}
\begin{document}
\title{\textbf{CIRCLE}}
\date{}
\maketitle

\begin{enumerate}
	\item In the given figure~\ref{fig:1},the quadrilateral PQRS circumscribes a circle.Here PA+CS is equal to : \\

\begin{figure}[h]
	        \centering
		\input{image/fig1.tex}
		\caption{}
		\label{fig:1}
\end{figure}



 \begin{figure}[h]
	        \centering
	        \input{image/tabel1.tex}
 \end{figure}




\item In the given figure~\ref{fig 2},$\vec{O}$ is the center of the circle.$\overrightarrow{AB}$ and $\overrightarrow{AC}$ are tangents drawn to the circle from point $\vec{ A}$.If $\angle BAC=65^{\circ}$, then find the measure of $\angle BOC$.\\


	\begin{center}
\begin{figure}[h]
	        \centering
	        \input{image/fig2.tex}
		\caption{}
		\label{fig 2}
        \end{figure}
	\end{center}


\pagebreak

\item In the given figure~\ref{fig:3},$\vec{ O}$ is the centre of the circle and $\overrightarrow {QPR}$ is a tangent to it at$\vec{ P}$. Prove that $\angle QAP+ \angle APR$= $90^{\circ}$.\\
\begin{figure}[h]
	        \centering
	        \input{image/fig02.tex}
		\caption{}
		\label{fig:3}

        \end{figure}


\item In the given figure~\ref{fig:4} ,$\overrightarrow{PQ}$ is tangent to the circle centred at $\vec{ O}$.If $\angle AOB= 95^{\circ}$, then the measure of $\angle ABQ$ will be


\begin{figure}[h]
	        \centering
	        \input{image/fig3.tex}
		\caption{}
		\label{fig:4}
        \end{figure}


\begin{figure}[h]
	        \centering
	        \setlength{\tabcolsep}{80pt}
 \renewcommand{\arraystretch}{2}
 
  \begin{tabular}{l c}
       A)47.5$^{\circ}$ & B)42.5$^{\circ}$ \\
       C)85$^{\circ}$& D)95$^{\circ}$
  \end{tabular}

        \end{figure}


\pagebreak
\item
  \begin{enumerate}
	  \item Two tangents $\overrightarrow{TP}$ and $\overrightarrow{TQ}$ are drawn between to a circle with centre $\vec{O}$ from an external point $\vec{T}$ (Figure~\ref{fig:5}). Prove that $\angle PTQ = 2 \angle OPQ$.\\
\begin{figure}[h]
	        \centering
	        \input{image/fig5.tex}
		\caption{}
		\label{fig:5}
        \end{figure}



\begin{center}
    \title{OR}
\end{center}
\item In the given figure~\ref{fig:6},a circle is inscribed in a quadrilateral ABCD in which $\angle B =90 ^{\circ}$.If AD=17cm,AB=20cm and DS=3cm, then find the radius of the circle.

\begin{figure}[h]
	        \centering
	        \input{image/fig6.tex}
		\caption{}
		\label{fig:6}
\end{figure}
   \end{enumerate}




\item The discus throw is an event in which an athlete attempts to throw a discus (as shown in the given figure~\ref{fig:0}).The athlete spins anti-clockwise around one and a half times through a circle, then releases the throw. When released, the discus travels along tangent to the circular spin orbit.


\begin{figure}[ht]	

	        \centering
		\includegraphics[width=0.7\columnwidth]{image/fig0.png}
		\caption{}
		\label{fig:0}
\end{figure}
\pagebreak

In the given figure~\ref{fig:7}, $\overrightarrow{AB}$ is one such tangent to a circle of radius 75 cm.Point $\vec{ O}$ is centre of the circle and $\angle ABO= 30^{\circ}$.PQ is parallel to OA.



\begin{figure}[ht]
	        \centering
	        \input{image/fig7.tex}
		\caption{}
		\label{fig:7}
        \end{figure}

     
Based on above information:

           \begin{enumerate}
		   \item find the length of $\overrightarrow{AB}$.
		   \item find the length of OB.
		   \item find the length of AP.
         \end{enumerate}
\begin{center}
\title{OR}
\end{center}
find the length of PQ.\\

\item In the given figure~\ref{fig:8},$\overrightarrow{ TA}$ is a tangent to the circle with centre $\vec{O}$ such that OT=4cm, $\angle OTA= 30 ^{\circ}$, then length of $\overrightarrow{TA}$ is:
      \begin{enumerate}
          \item $2\sqrt3 cm$
          \item 2 cm
          \item $2\sqrt2$ cm
          \item $\sqrt3$ cm
      \end{enumerate}




  \begin{figure}[h]
	\centering
	\input{image/fig8.tex}
	\caption{}
	\label{fig:8}
  \end{figure}
\pagebreak



\item In the given figure~\ref{fig:9},$\overrightarrow{PT}$ is a tangent at $\vec{T}$ to the circle with centre $\vec{O}$.If $\angle TPO=25^{\circ}$, then \textit{x} is equal to:
   \begin{enumerate}
       \item 25$^{\circ}$
       \item 65$^{\circ}$  
       \item 90$^{\circ}$
       \item 115$^{\circ}$
       
   \end{enumerate}




\begin{figure}[h]
	        \centering
	        \input{image/fig9.tex}
		\caption{}
		\label{fig:9}
\end{figure}
\pagebreak



\item Two concentric circles are of radii 5 cm and 3 cm.Find the length of the cord of the larger circle which touches the smaller circle.

         
\end{enumerate}

\end{document}
