\documentclass[12pt]{article}
\usepackage{float}
\usepackage{gensymb}
\usepackage{amsmath}
\usepackage{graphics}                         
\usepackage{graphicx}                     
\graphicspath{{storage/self/primary/Download/asgnt6/fig}}                            
\graphicspath{{storage/self/primary/Download/asgnt6/table}}
\providecommand{\brak}[1]{\ensuremath{\left(#1\right)}}
\providecommand{\myvec}[1]{\ensuremath{\begin{pmatrix}#1\end{pmatrix}}}
\providecommand{\norm}[1]{\ensuremath{\lvert|#1\rvert|}}
\let\vec\mathbf
\begin{document}
\title{\textbf{9.10.4.3}}
\date{}
\maketitle
\textbf{Question :} If two equal chords of a circle intersect within the circle,prove that the line joining the point of intersection to the centre makes equal angles with the chords.

\textbf{Solution :}

\begin{table}[H]
    \centering
    \begin{tabular}{|c|c|c|}
 \hline
    \textbf{Input Parameters} &\textbf{Description} &\textbf{Value} \\
    \hline
     $\vec{A}$& Vertex(at origin)&$\vec{0}$\\
     \hline
	$a$& Side of the parallelogram &$AB \brak{= DC = 6 unit}$\\
     \hline
	$b$& Side of the parallelogram &$AD\brak{= BC = \sqrt{29} unit}$\\
     \hline
	$\theta$&Angle of parallelogram & $\angle BAD\brak{= \sin^{-1}\brak{\frac{5}{\sqrt{29}}}}$\\
     \hline
      $k_1:1$&Ratio by which $\vec{Q}$ divides $AD$  & $AQ:QD$\\
     \hline  
      $k_2:1$&Ratio by which $\vec{P}$ divides $DC$  & $DP:PC$\\
     \hline  
\end{tabular}

    \caption{Table of input parameters}
    \label{tab:tab:1}
\end{table}

\begin{table}[]
    \centering
\begin{tabular}{|c|c|}
\hline
    \textbf{Output Parameters} &\textbf{Value} \\
    \hline
     $\vec{B}$&$a\begin{pmatrix}
         1\\0
     \end{pmatrix}$\\
     \hline
     $\vec{D}$&$b\begin{pmatrix}
         \cos{\theta}\\\sin{\theta}
     \end{pmatrix}$\\
     \hline
     $\vec{C}$&$\vec{B}+\vec{D}$\\
     \hline
      $\vec{ Q}$ & $\frac{k_1.\vec{D}+\vec{A}}{k_1+1}$\\
     \hline
      $\vec{P}$ &$\frac{k_2.\vec{C}+\vec{D}}{k_2+1}$\\
   \hline
\end{tabular}


\caption{Table of output parameters}
    \label{tab:tab:2}
\end{table}

\begin{align}
  \cos{\angle RQS}&=\frac{\vec{\brak{R-Q}}^{\top}\vec{\brak{Q-S}}}{\vec{\norm{R-Q}\norm{Q-S}}}\\
  &=\cos{\frac{\theta_3-\theta_4}{2}}\\
   \cos{\angle PSQ}&=\frac{\vec{\brak{P-S}}^{\top}\vec{\brak{S-Q}}}{\vec{\norm{P-S}\norm{S-Q}}}\\
  &=\cos{\frac{\theta_1-\theta_2}{2}}\\
   \cos{\angle RQS}&=\cos{\angle PSQ}\\
   or,\theta_1&=\theta_2-\theta_3+\theta_4\\
   &=-180\degree
  \end{align}
The equation of $PQ$ and $RS$ is obtained by
\begin{align}
  \vec{n_1}^{\top}\vec{\brak{x-P}}=0\\
  \vec{n_2}^{\top}\vec{\brak{x-R}}=0\\
  or,\vec{n_1}=\myvec{\sin{\theta_2}-\sin{\theta_1}\\\cos{\theta_1}-\cos{\theta_2}}\\
  or,\vec{n_2}=\myvec{\sin{\theta_4}-\sin{\theta_3}\\\cos{\theta_3}-\cos{\theta_4}}\\
 \end{align}
  The value of the point of the intersection is
  \begin{align}
      \vec{T} &= \myvec{0\\-0.45}\\
\cos{\angle OTP} &= \frac{\brak{\vec{T-P}}^{\top}\brak{\vec{O-T}}}{\vec{\norm{T-P}\norm{T-O}}}\\
or, \angle OTP&= 65.8\degree
\end{align}
Similarly,
\begin{align}
   \cos{\angle OTR} &= \frac{\brak{\vec{T-R}}^{\top}\brak{\vec{T-O}}}{\vec{\norm{T-R}\norm{T-O}}}\\
or, \angle OTR&= 65.8\degree \\
So,\angle OTP &= \angle OTR \brak{proved}
\end{align}

  

\begin{figure}[H]                             
	\centering
	\includegraphics[width=\columnwidth]{fig/9.10.4.3.png}                            
	\caption{}                              
	\label{fig:9.10.4.3}
\end{figure}
\end{document}
