\documentclass[12pt]{article}
\usepackage{refstyle}
\usepackage{hyperref}
\usepackage{amsmath}
\usepackage{gensymb}
\usepackage{graphics}
\usepackage{graphicx}
\newcommand\norm[1]{\left\Vert#1\right\Vert}
\usepackage{float}
\providecommand{\brak}[1]{\ensuremath{\left(#1\right)}}
\providecommand{\myvec}[1]{\ensuremath{\begin{pmatrix}#1\end{pmatrix}}}
\providecommand{\innpdt}[1]{\ensuremath{\langle#1\rangle}}

\let\vec\mathbf
\graphicspath{{storage/self/primary/Download/embedded/fig}}
\graphicspath{{storage/self/primary/Download/embedded/table}}
\begin{document}
\title{\textbf{EMBEDDED}}
\date{}
\maketitle
\textbf{Question :} $l$ and $m$ are two parallel lines intersected by another pair of parallel lines $p$ and $q (\figref{fig:1})$,show that $\triangle ABC \cong \triangle CDA$.

\textbf{Figure :}
\begin{figure}[H]
    \centering
    \includegraphics{fig/em1.png}
    \caption{Required parallelogram}
    \label{fig:fig:1}
\end{figure}

\textbf{Solution :}
\begin{table}[H]
    \centering
    \begin{tabular}{|c|c|c|}
 \hline
    \textbf{Input Parameters} &\textbf{Description} &\textbf{Value} \\
    \hline
     $\vec{A}$& Vertex(at origin)&$\vec{0}$\\
     \hline
	$a$& Side of the parallelogram &$AB \brak{= DC = 6 unit}$\\
     \hline
	$b$& Side of the parallelogram &$AD\brak{= BC = \sqrt{29} unit}$\\
     \hline
	$\theta$&Angle of parallelogram & $\angle BAD\brak{= \sin^{-1}\brak{\frac{5}{\sqrt{29}}}}$\\
     \hline
      $k_1:1$&Ratio by which $\vec{Q}$ divides $AD$  & $AQ:QD$\\
     \hline  
      $k_2:1$&Ratio by which $\vec{P}$ divides $DC$  & $DP:PC$\\
     \hline  
\end{tabular}

    \caption{Table of input parameters}
    \label{tab:tab:1}
\end{table}

\begin{table}[H]
    \centering
    \begin{tabular}{|c|c|}
\hline
    \textbf{Output Parameters} &\textbf{Value} \\
    \hline
     $\vec{B}$&$a\begin{pmatrix}
         1\\0
     \end{pmatrix}$\\
     \hline
     $\vec{D}$&$b\begin{pmatrix}
         \cos{\theta}\\\sin{\theta}
     \end{pmatrix}$\\
     \hline
     $\vec{C}$&$\vec{B}+\vec{D}$\\
     \hline
      $\vec{ Q}$ & $\frac{k_1.\vec{D}+\vec{A}}{k_1+1}$\\
     \hline
      $\vec{P}$ &$\frac{k_2.\vec{C}+\vec{D}}{k_2+1}$\\
   \hline
\end{tabular}


    \caption{Table of output parameters}
    \label{tab:tab:2}
\end{table}  
From $\figref{fig:1}$ between $\triangle ABC $ and $\triangle CDA$
\begin{align}
\cos{\angle BAC} &= \vec{\frac{\innpdt{B-A,A-C}}{\norm{B-A}\norm{A-C}}}\\
&= \frac{\brak{\vec{B-A}}^T\vec{\brak{A-C}}}{\vec{\norm{B-A}\norm{A-C}}}\\
&=\frac{ab\cos{\theta}+b\sin{\theta}}{b\sqrt{a^2-2ab\cos{\theta}+b^2}}\\
&=\frac{17}{\sqrt{29}\sqrt{41}}\\
\cos{\angle ACD} &= \vec{\frac{\innpdt{D-C,A-C}}{\norm{D-C}\norm{A-C}}}\\
&= \frac{\brak{\vec{D-C}}^T\vec{\brak{A-C}}}{\vec{\norm{D-C}\norm{A-C}}}\\
&= \frac{b^2-ab\cos{\theta}}{b\sqrt{a^2-2ab\cos{\theta}+b^2}}\\
&=\frac{17}{\sqrt{29}\sqrt{41}}\\
So,\angle BAC = \angle ACD.\\
\cos{\angle ACB} &= \vec{\frac{\innpdt{B-C,A-C}}{\norm{B-C}\norm{A-C}}}\\
&= \frac{\brak{\vec{B-C}}^T\vec{\brak{A-C}}}{\vec{\norm{B-C}\norm{A-C}}}\\
&=\frac{a^2-ab\cos{\theta}}{a\sqrt{a^2-2ab\cos{\theta}+b^2}}\\
&=\frac{24}{6\sqrt{41}}\\
\cos{\angle} CAD &= \vec{\frac{\innpdt{A-D,A-C}}{\norm{A-D}\norm{A-C}}}\\
&= \frac{\brak{\vec{A-D}}^T\vec{\brak{A-C}}}{\vec{\norm{A-D}\norm{A-C}}}\\
&=\frac{a^2-ab\cos{\theta}}{a\sqrt{a^2-2ab\cos{\theta}+b^2}}\\
&=\frac{24}{6\sqrt{41}}\\
So,\angle ACB = \angle CAD.
\end{align}
And $CA$ is common side .

So,$\triangle ABC \cong \triangle CDA.\brak{by A-A-S}\brak{proved}$
\end{document}

