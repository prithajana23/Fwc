\documentclass[12pt]{article}
\usepackage{gensymb}
\usepackage{amsmath}
\usepackage{graphics}
\usepackage{hyperref}
\usepackage{refstyle}
\usepackage{graphicx}
\graphicspath{{storage/self/primary/Download/asgnt5/fig}}
\graphicspath{{storage/self/primary/Download/asgnt5/table}}
\let\vec\mathbf
\usepackage{float}
\providecommand{\brak}[1]{\ensuremath{\left(#1\right)}}
\providecommand{\myvec}[1]{\ensuremath{\begin{pmatrix}#1\end{pmatrix}}}
\providecommand{\norm}[1]{\ensuremath{\lvert|#1\rvert|}}
\begin{document}
\title{\textbf{9.10.5.3}}
\date{}
\maketitle
\textbf{Question :} In \figref{fig:9.10.5.3.1} or,\figref{fig:9.10.5.3.2}$\angle PQR = 100\degree$,where $P,Q$ and $R$ are points on a circle with centre $O$.Find $\angle OPR$.

\textbf{Solution :}
\begin{table}[H]
    \centering
    \begin{tabular}{|c|c|c|}
 \hline
    \textbf{Input Parameters} &\textbf{Description} &\textbf{Value} \\
    \hline
     $\vec{A}$& Vertex(at origin)&$\vec{0}$\\
     \hline
	$a$& Side of the parallelogram &$AB \brak{= DC = 6 unit}$\\
     \hline
	$b$& Side of the parallelogram &$AD\brak{= BC = \sqrt{29} unit}$\\
     \hline
	$\theta$&Angle of parallelogram & $\angle BAD\brak{= \sin^{-1}\brak{\frac{5}{\sqrt{29}}}}$\\
     \hline
      $k_1:1$&Ratio by which $\vec{Q}$ divides $AD$  & $AQ:QD$\\
     \hline  
      $k_2:1$&Ratio by which $\vec{P}$ divides $DC$  & $DP:PC$\\
     \hline  
\end{tabular}

    \caption{Table of input parameters}
    \label{tab:tab:1}
\end{table}

\begin{table}[H]
    \centering
    \begin{tabular}{|c|c|}
\hline
    \textbf{Output Parameters} &\textbf{Value} \\
    \hline
     $\vec{B}$&$a\begin{pmatrix}
         1\\0
     \end{pmatrix}$\\
     \hline
     $\vec{D}$&$b\begin{pmatrix}
         \cos{\theta}\\\sin{\theta}
     \end{pmatrix}$\\
     \hline
     $\vec{C}$&$\vec{B}+\vec{D}$\\
     \hline
      $\vec{ Q}$ & $\frac{k_1.\vec{D}+\vec{A}}{k_1+1}$\\
     \hline
      $\vec{P}$ &$\frac{k_2.\vec{C}+\vec{D}}{k_2+1}$\\
   \hline
\end{tabular}


    \caption{Table of output parameters}
    \label{tab:tab:2}
\end{table}

For getting the value of the $\angle NOQ$
\begin{align}
    \cos{\theta}&=\frac{\brak{\vec{R-Q}}^{\top}\brak{\vec{P-Q}}}{\vec{\norm{R-Q}\norm{P-Q}}}\\
    or,\cos{\theta}&=\frac{\sin{\frac{\theta_1+\theta_2}{2}}\cos{\frac{\theta_2+\theta_3}{2}}}{\sin{\frac{\theta_2-\theta_1}{2}}}\\
    so,\theta_1&=2\tan^{-1}\brak{\tan\brak{{\frac{\theta_2}{2}}}\brak{\frac{\cos{\theta}+\cos{\frac{\theta_2+\theta_3}{2}}}{\cos{\theta}-\cos{\frac{\theta_2+\theta_3}{2}}}}}\\
    \implies \theta_1&=136.696\degree\\
    or,\theta_1&=2\tan^{-1}\brak{\tan\brak{{\frac{\theta_2}{2}}}\brak{\frac{\cos{\theta}-\cos{\frac{\theta_2+\theta_3}{2}}}{\cos{\theta}+\cos{\frac{\theta_2+\theta_3}{2}}}}}\\
    \implies \theta_1&=175\degree\\
\end{align}
For getting the value of the $\angle OPR$
\begin{align}
    \angle POR &=360\degree -2\angle PQR\\
    &=360\degree -2\theta \\
    \angle POR +\angle ORP +\angle OPR &= 180\degree\\
    \angle POR +2\angle OPR&= 180\degree, \brak{OR=OP}\\
\angle OPR &= \frac{2\theta-180\degree}{2}\\
 \angle OPR &=10\degree
 \end{align}

  \begin{figure}[H]                                
 \centering                          
 \includegraphics[width=\columnwidth]{fig/9.10.5.3.1.png}
\caption{}
\label{fig:fig:9.10.5.3.1}                        
  \end{figure}
   \begin{figure}[H]                                
 \centering                          
 \includegraphics[width=\columnwidth]{fig/9.10.5.3.2.png}
\caption{}
\label{fig:fig:9.10.5.3.2}                        
  \end{figure}

\end{document}

