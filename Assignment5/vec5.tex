\documentclass[12pt]{article}
\usepackage{gensymb}
\usepackage{amsmath}
\usepackage{graphics}
\usepackage{graphicx}
\graphicspath{{storage/self/primary/Download/asgnt5/fig}}
\graphicspath{{storage/self/primary/Download/asgnt5/table}}
\let\vec\mathbf
\usepackage{float}
\providecommand{\brak}[1]{\ensuremath{\left(#1\right)}}
\providecommand{\myvec}[1]{\ensuremath{\begin{pmatrix}#1\end{pmatrix}}}
\providecommand{\norm}[1]{\ensuremath{\lvert|#1\rvert|}}
\begin{document}
\title{\textbf{9.10.5.3}}
\date{}
\maketitle
\textbf{Question :} In  ,$\angle PQR = 100\degree$,where $P,Q$ and $R$ are points on a circle with centre $O$.Find $\angle OPR$.

\textbf{Solution :}
\begin{table}[H]
    \centering
    \begin{tabular}{|c|c|c|}
 \hline
    \textbf{Input Parameters} &\textbf{Description} &\textbf{Value} \\
    \hline
     $\vec{A}$& Vertex(at origin)&$\vec{0}$\\
     \hline
	$a$& Side of the parallelogram &$AB \brak{= DC = 6 unit}$\\
     \hline
	$b$& Side of the parallelogram &$AD\brak{= BC = \sqrt{29} unit}$\\
     \hline
	$\theta$&Angle of parallelogram & $\angle BAD\brak{= \sin^{-1}\brak{\frac{5}{\sqrt{29}}}}$\\
     \hline
      $k_1:1$&Ratio by which $\vec{Q}$ divides $AD$  & $AQ:QD$\\
     \hline  
      $k_2:1$&Ratio by which $\vec{P}$ divides $DC$  & $DP:PC$\\
     \hline  
\end{tabular}

    \caption{Table of input parameters}
    \label{tab:tab:1}
\end{table}

\begin{table}[H]
    \centering
    \begin{tabular}{|c|c|}
\hline
    \textbf{Output Parameters} &\textbf{Value} \\
    \hline
     $\vec{B}$&$a\begin{pmatrix}
         1\\0
     \end{pmatrix}$\\
     \hline
     $\vec{D}$&$b\begin{pmatrix}
         \cos{\theta}\\\sin{\theta}
     \end{pmatrix}$\\
     \hline
     $\vec{C}$&$\vec{B}+\vec{D}$\\
     \hline
      $\vec{ Q}$ & $\frac{k_1.\vec{D}+\vec{A}}{k_1+1}$\\
     \hline
      $\vec{P}$ &$\frac{k_2.\vec{C}+\vec{D}}{k_2+1}$\\
   \hline
\end{tabular}


    \caption{Table of output parameters}
    \label{tab:tab:2}
\end{table}
For getting the value of the $\angle OPR$
\begin{align}
    \cos{\angle OPR }&= \frac{\vec{\brak{O-P}}^{\top}\vec{\brak{R-P}}}{\vec{\norm{O-P}\norm{R-P}}}\\
    \angle OPR &= 10\degree
  \end{align}

  \begin{figure}[H]                                 
	  \centering                          
	  \includegraphics[width=\columnwidth]{fig/9.10.5.3.png}
\caption{}
\label{9.10.5.3}                        
  \end{figure}

\end{document}

